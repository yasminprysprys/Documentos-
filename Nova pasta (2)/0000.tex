\documentclass[a4paper]{article}
\usepackage{calc,amsmath,amssymb,amsfonts}
\usepackage[T1]{fontenc}
\usepackage[english,portuges]{babel}
\usepackage{xcolor,fancyhdr}
\usepackage[vmargin=0.9839in,hmargin=1.1811in,noheadfoot]{geometry}
\usepackage{enumitem,hyperref}
\hypersetup{colorlinks=true,allcolors=blue,pdfauthor=Almir Evangelista}
% Outline numbering
\setcounter{secnumdepth}{0}
% Text styles
\newcommand\textstyleListLabeli[1]{\textrm{#1}}
\newcommand\textstyleListLabelx[1]{\textrm{#1}}
\newcommand\textstyleListLabelxix[1]{\textrm{#1}}
\newcommand\textstyleListLabelxxviii[1]{\textrm{#1}}
% Pages
\fancypagestyle{Standard}{\fancyhf{}
  \fancyhead[L]{}
  \fancyfoot[L]{}
  \renewcommand\headrulewidth{0pt}
  \renewcommand\footrulewidth{0pt}
  \renewcommand\thepage{\arabic{page}}
}
\pagestyle{Standard}
\author{Almir Evangelista}
\date{2024-04-08}
\begin{document}
\clearpage
\pagestyle{Standard}
\section{Documentação Projeto Integrador 5}

\bigskip

\textbf{Objetivo: }

Desenvolver dashboards com dados do censo da Educação Superior a fim de trazer visualização clara de informações de
todos os cursos de graduação e trabalhar em cima das mesmas.


\bigskip

\textbf{Equipe e divisão: }

\textbf{Líder: }Almir Evangelista

\textbf{Desenvolvedores: }Almir Evangelista e Daniel Vitor

\textbf{Designers: }Júlio Vitor e Yasmin Priscila

\textbf{Especialistas no problema:} Letícia Vitória e Lucas Medeiros


\bigskip

\textbf{Problema: }

Falta de uma visualização geral e organizada das informações.

\textcolor[HTML]{1F1F1F}{A base de dados que possuímos contém uma grande quantidade de dados, como:}

\begin{itemize}[series=listWWNumi,label=\textstyleListLabeli{[F0B7?]}]
\item \textbf{\textcolor[HTML]{1F1F1F}{Nomes dos cursos}}
\item \textbf{\textcolor[HTML]{1F1F1F}{Instituições que os oferecem}}
\item \textbf{\textcolor[HTML]{1F1F1F}{Duração}}
\item \textbf{\textcolor[HTML]{1F1F1F}{Mensalidades}}
\item \textbf{\textcolor[HTML]{1F1F1F}{Localização}}
\item \textbf{\textcolor[HTML]{1F1F1F}{Grade curricular}}
\item \textbf{\textcolor[HTML]{1F1F1F}{Estudantes matriculados}}
\item \textbf{\textcolor[HTML]{1F1F1F}{Professores}}
\item \textbf{\textcolor[HTML]{1F1F1F}{Avaliações de cursos}}
\item \textbf{\textcolor[HTML]{1F1F1F}{Mercado de trabalho para cada área}}
\end{itemize}
\textcolor[HTML]{1F1F1F}{Embora esses dados sejam valiosos, eles podem ser difíceis de analisar e interpretar em seu
formato bruto. É aí que os dashboards entram em ação.}

\textbf{\textcolor[HTML]{1F1F1F}{Os dashboards permitem que você:}}

\begin{itemize}[series=listWWNumii,label=\textstyleListLabelx{[F0B7?]}]
\item \textbf{\textcolor[HTML]{1F1F1F}{Visualize os dados de forma clara e concisa}}
\item \textbf{\textcolor[HTML]{1F1F1F}{Identifique tendências e padrões}}
\item \textbf{\textcolor[HTML]{1F1F1F}{Compare diferentes cursos e instituições}}
\item \textbf{\textcolor[HTML]{1F1F1F}{Faça perguntas e encontre respostas}}
\item \textbf{\textcolor[HTML]{1F1F1F}{Tome decisões mais informadas}}
\end{itemize}
\textcolor[HTML]{1F1F1F}{Por exemplo, você pode criar um dashboard que mostre a
}\textbf{\textcolor[HTML]{1F1F1F}{distribuição dos cursos por área de estudo}}\textcolor[HTML]{1F1F1F}{, o
}\textbf{\textcolor[HTML]{1F1F1F}{custo médio de cada curso}}\textcolor[HTML]{1F1F1F}{, a
}\textbf{\textcolor[HTML]{1F1F1F}{satisfação dos alunos com diferentes cursos}}\textcolor[HTML]{1F1F1F}{, ou a
}\textbf{\textcolor[HTML]{1F1F1F}{demanda por profissionais em diferentes áreas}}\textcolor[HTML]{1F1F1F}{.}

\textcolor[HTML]{1F1F1F}{Com esses dashboards, você pode ajudar os alunos a:}

\begin{itemize}[series=listWWNumiii,label=\textstyleListLabelxix{[F0B7?]}]
\item \textbf{\textcolor[HTML]{1F1F1F}{Escolher o curso superior mais adequado para seus objetivos}}
\item \textbf{\textcolor[HTML]{1F1F1F}{Comparar diferentes opções de cursos e instituições}}
\item \textbf{\textcolor[HTML]{1F1F1F}{Tomar decisões mais informadas sobre seu futuro profissional}}
\end{itemize}
\textcolor[HTML]{1F1F1F}{Além disso, os dashboards podem ser usados por:}

\begin{itemize}[series=listWWNumiv,label=\textstyleListLabelxxviii{[F0B7?]}]
\item \textbf{\textcolor[HTML]{1F1F1F}{Instituições de ensino superior para avaliar seus cursos e identificar áreas de
melhoria}}
\item \textbf{\textcolor[HTML]{1F1F1F}{Governo para formular políticas públicas de educação}}
\item \textbf{\textcolor[HTML]{1F1F1F}{Empresas para identificar oportunidades de recrutamento}}
\end{itemize}
\textcolor[HTML]{1F1F1F}{Em resumo, os dashboards que você está criando podem ser uma ferramenta poderosa para
}\textbf{\textcolor[HTML]{1F1F1F}{melhorar a compreensão e o uso dos dados sobre cursos
superiores}}\textcolor[HTML]{1F1F1F}{.}

\textbf{Cliente: }

Universidade Católica de Pernambuco (UNICAP)


\bigskip

\textbf{Usuários:}

Gestores da Universidade Católica de Pernambuco (UNICAP)


\bigskip

\newline
\textbf{Localização do problema:}

O problema da falta de uma visão geral e organizada das informações sobre cursos superiores se manifesta em diversos
ambientes:

\textbf{1. Instituições de ensino superior:}

Dificuldade em acompanhar a demanda por informações por parte dos alunos.

Falta de dados para embasar decisões estratégicas sobre os cursos.

Dificuldade em avaliar a efetividade dos cursos.

\textbf{2. Estudantes:}

Dificuldade em encontrar informações confiáveis e comparáveis sobre cursos.

Dificuldade em tomar decisões informadas sobre sua carreira.

Possibilidade de escolher um curso que não seja adequado para seus objetivos.


\bigskip

\textbf{3. Governo:}

Dificuldade em formular políticas públicas de educação eficazes.

Falta de dados para avaliar o impacto dos investimentos em educação superior.

\textbf{4. Empresas:}

Dificuldade em encontrar profissionais qualificados para as suas vagas.

Falta de dados para identificar as habilidades necessárias para o futuro mercado de trabalho.

\textbf{Impacto do problema:}

O impacto da falta de uma visão geral e organizada das informações sobre cursos superiores é significativo e abrangente:

\textbf{Pessoas:}

\textbf{Frustração e desânimo:} Estudantes podem se sentir perdidos e desorientados na hora de escolher um curso
superior.

\textbf{Decisões equivocadas:} A escolha de um curso inadequado pode ter um impacto negativo na vida profissional e
pessoal do estudante.

\textbf{Desperdício de tempo e recursos:} Estudantes podem investir tempo e dinheiro em cursos que não atendem às suas
expectativas.

\textbf{Impacto financeiro:}

\textbf{Perda de receita para as instituições de ensino superior:} A falta de informações claras e concisas pode levar à
perda de alunos.

\textbf{Aumento dos custos para o governo:} O governo pode ter que investir mais em programas de orientação profissional
para ajudar os alunos a escolherem o curso certo.

\textbf{Prejuízo para as empresas:} As empresas podem ter que investir mais em treinamento e desenvolvimento de seus
funcionários.

\textbf{Abrangência:}

\textbf{Problema global:} A falta de acesso a informações sobre cursos superiores é um problema que afeta países de todo
o mundo.

\textbf{Impacto em milhões de pessoas:} Todos os anos, milhões de pessoas tomam decisões sobre seu futuro profissional
sem ter acesso a informações completas e confiáveis.

\textbf{Problema crescente:} A demanda por informações sobre cursos superiores está aumentando, e a falta de uma solução
adequada pode ter um impacto ainda maior no futuro.


\bigskip


\bigskip


\bigskip

\textbf{Concorrentes:}

\textbf{1. Plataformas de busca de cursos:}

\begin{itemize}[series=listWWNumxii,label=[F0B7?]]
\item Exemplos: Quero Bolsa, Guia da Faculdade, Estudar Fora.
\item Funcionalidades: Busca por cursos, comparação de cursos, avaliações de cursos, informações sobre instituições.
\item Vantagens: Ampla variedade de cursos e instituições, informações detalhadas sobre cursos, avaliações de alunos.
\item Desvantagens: Falta de personalização, dificuldade em comparar cursos de diferentes plataformas, foco em cursos
presenciais.
\end{itemize}
\textbf{2. Sites de instituições de ensino superior:}

\begin{itemize}[series=listWWNumxiii,label=[F0B7?]]
\item Funcionalidades: Informações sobre cursos, grade curricular, corpo docente, vestibular etc.
\item Vantagens: Informações oficiais e atualizadas sobre a instituição, possibilidade de contato direto com a
instituição.
\item Desvantagens: Falta de padronização nas informações, dificuldade em comparar cursos de diferentes instituições,
foco em informações institucionais.
\end{itemize}

\bigskip


\bigskip
\end{document}
